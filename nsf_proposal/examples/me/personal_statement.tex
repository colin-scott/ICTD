\documentclass[12pt]{article}	% if curious, google for different types of document classes
				% a resume class also exists and makes for a nice-looking resume

\usepackage[top=1in,right=1in,left=1in,bottom=1in]{geometry}
\pagestyle{empty} 		% use if page numbers not wanted
\usepackage{verbatim}

\newenvironment{packed_enum}{
\begin{enumerate}
  \setlength{\itemsep}{1pt}
  \setlength{\parskip}{0pt}
  \setlength{\parsep}{0pt}
}{\end{enumerate}}

%%%% Beginning of the Document %%%%
\begin{document}

\begin{center}
{ \bf Personal Statement } 
\end{center}

%"A Ph.D. requires a particular type of personality. You need to be someone who
%is obsessed with figuring out a problem. You need to have tremendous perseverance and be
%capable of hard work. You need to be willing to do whatever it takes to solve your %problem (e.g., take 5 math classes, learn a whole new area like databases, rewrite the whole kernel, %etc.)."
%
%"You need to know why you want a Ph.D. You need to have vision and ideas and
%you need to be able to express yourself.∗"

My initial motivation for pursuing computer science was simple
curiosity; I wanted to understand how computer systems work ``under-the-hood''.
My exposure to research has since caused me to focus on different
questions: how {\em should} computers systems work?, {\em why} did we
design them the way we did?, what do we not yet know about their behavior and
limitations?, what problems remain unsolved?, and how do we approach them?
Questions of this sort are appealing to me, both because they are technically
challenging, and because they affect our interactions with computer systems on
a daily basis. The excitement of developing answers to them and communicating
the resulting knowledge drives my desire to pursue 
graduate school.

I was drawn to the reverse traceroute project by the prospect of discovering
unknown properties of a large and complex system: the Internet.
Networks research is especially interesting to me because it involves the
interactions of many different human and system components. My experience working on reverse traceoute has left me with an understanding
of the research process, both as an individual and as part of a larger research agenda.
But the lessons I have learned are more than intellectual;
the direct applicability of reverse traceroute to problems affecting users of the Internet all over the world
has undoubtedly shaped my research interests.

This summer I had the opportunity to apply the knowledge I had learned from
working on reverse traceroute at an internship with Amazon Web Service's content distribution network.
Being exposed to real-world operational issues solidified my interest in
networks research. The Internet is plagued with problems, and researchers are
in an excellent position to develop solutions. Working in industry also helped me understand the factors companies must consider in adopting
a new piece of infrastructure. Because adoption is so crucial to having 
impact, I keep these considerations in mind when pursuing my own research interests.

%Indeed, innovations have no value until they change the way we
%live and think.
Sharing knowledge with others is just as important as
discovering it. I've enjoyed opportunities to
teach and mentor throughout my undergraduate career.

Up until my senior year, I was never really exposed to situations where I
needed to argue for the value of my research to others. I have
only recently come to realize the huge importance of this skill. Seeing that
many of my fellow undergraduate researchers also lacked experience
communicating their work, I helped organize a weekly meeting where students
take turns presenting and leading discussions on their current projects.
I hope to expand this opportunity to a wider audience, especially 
undergraduates looking to begin research; the best way to imbue otherwise
shy students with the confidence to pursue research is by showing them
first hand the scope and relevance of research carried out by their own peers.

Participating in research is easily the most rewarding experience I have
had as an undergraduate, and I strongly enjoy encouraging other students to
give it a try. For the past two years I have served as a panelist at
undergraduate research night, where I shared my experiences working with
faculty and graduate students to an audience of peers interested in pursuing research.

%In addition to computer science, I have also pursued an interest in
%philosophy. Many of my

My goals in graduate school are to develop techniques for improving our understanding of the behavior of computer
networks, and to use the knowledge gained to develop solutions to problems that have
immediate impact to users. My previous research and internship put me in a position
to achieve these goals, and the NSF fellowship will give me the
freedom to pursue them.

\end{document}
