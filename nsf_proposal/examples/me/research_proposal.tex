\documentclass[12pt]{article}	% if curious, google for different types of document classes
				% a resume class also exists and makes for a nice-looking resume
\usepackage[top=1in,right=1in,left=1in,bottom=1in]{geometry}
\pagestyle{empty} 		% use if page numbers not wanted
\usepackage{verbatim}

\newenvironment{packed_enum}{
\begin{enumerate}
  \setlength{\itemsep}{1pt}
  \setlength{\parskip}{0pt}
  \setlength{\parsep}{0pt}
}{\end{enumerate}}

%%%% Beginning of the Document %%%%
\begin{document}

\begin{center}
{\bf Proposed Plan of Research } 
\end{center}

\noindent{\bf Title}: Towards a negotiation framework for improving interdomain routing  \\
\noindent{\bf Keywords}: protocol design, Internet measurement, interdomain routing

{\em I propose to investigate extensions to the border gateway protocol (BGP)
for facilitating negotiation of optimal and mutually beneficial routes between
 autonomous systems (ASes). This work would impact nearly every application that
makes use of the Internet. }

\noindent{\bf Background} \\
\indent The Internet plays a central role in our society. We use it for
communication, business transactions, education, entertainment, news, and
much more. Given our reliance on the Internet, it is crucial to ensure that network
connections have both high reliability and low latency. For some
applications, the reliability and performance of network connections is absolutely necessary.
For example, consider a doctor's use of the Internet for remotely diagnosing and
treating clients, or the military's need for real-time coordination of
resources.

Somewhat surprisingly, routes on the Internet are often highly inefficient
[1]. This is partly due to the fact that the Internet is controlled by many
independent organizations all acting in their own self-interest; although
these organizations must ultimately cooperate to achieve their goals, they
often make conflicting policy decisions which hurt the efficiency and
stability of the overall system.

To take a concrete example, consider a content provider, say Netflix, and a
distant stub AS forwarding requests towards Netflix's prefix. From
Netflix's perspective, optimizing latency is highly important. However, it may be the
case that the stub AS is using a circuitous route to forward traffic, perhaps
because they prefer routes advertised by their cheapest provider, or because
their border router is misconfigured. In this case, Netflix has very little control
over the routes chosen by the stub AS, even though a mutually beneficial
alternative may exist.

%Misconfigurations are particularly compelling
%since it is clearly in both ASes interest to use different routes.
There are three key issues involved: first, it is
difficult for network operators to infer which routes other ASes are
using, or whether they have alternative path available to them. Second, once
an operator has identified a problem, they lack straightforward methods of affecting the routes used
by of other ASes, especially those which are not directly adjacent. Current
practice involves the use of ad-hoc techniques such as AS prepending to
induce a change in route preference. These techniques are generally both coarse-grained
and unpredictable. If they do not prevail, operators may be forced to contact the
perpetrating AS directly. Third, reaching mutually beneficial routing
arrangements requires the sharing of potentially sensitive or propietery information
between ASes; ASes hide information that might give others a
competitive advantage, such as business relationships or traffic costs, but at
the same time they may need to share some amount of information in order to improve
network performance or reliability.

These issues partly stem from properties of BGP, the protocol used for negotiating inter-domain routes.
In particular, BGP i.) only allows exchange of routing
information with directly adjacent neighbors, and ii.) only facilitates sharing
of a single ``preferred'' route for any given prefix, even though alternate paths may
be available.

\noindent{\bf Proposed Research} \\
% afflicted with many problems, such as persistent oscillation, forwarding
% loops, slow convergence, and network partitions.
% I became aware of this problem during a summer internship at Amazon Web Services' content distribution
% network. 
\indent I propose to investigate extensions to BGP that would allow network
operators to reach mutually beneficial routing configurations with other
ASes in a general and straightforward way. My research will explore three
unsolved problems in this area: limited visibility, non-adjacent negotiation, and
performance negotiation.

MIRO [3] explores the possibility of multi-path interdomain routing: in
addition to existing BGP-learned routes, arbitrary pairs of ASes can negotiate further
alternate paths for certain traffic flows. One issue which is not addressed by MIRO is
that global AS topology is generally not known; in order to achieve their objectives,
operators first need know which ASes to negotiate with. Part of my work
will address the common case where operators have only partial visibility into the AS
topology. In order to hide business relationships and other propietery
information, I will evaluate recursive route negotiations -- delegating a request
to a neighbor -- and flooding protocols.

How will the negotiations be carried out? Given that ASes are in competition, the negotiation mechanism must ensure that
outcomes are mutually beneficial and that no unwanted information is leaked. Previous research [2]
investigated a framework for joint negotiation of BGP-learned routes between directly adjacent ASes by exposing only coarse,
opaque preferences. Can this approach be generalized to allow for negotiation between non-adjacent ASes? Because intermediate ASes are involved, a new
incentive model is required.

Because ASes often attempt to optimize performance, automated negotiation with
performance quantifiers would be extremely valuable. For example, an AS could
send a message saying "Does anyone have a route with X latency
to AS B"?. Neither MIRO nor Nexit explore this
possibility. A pricing and bidding mechanism would be required to incentivize
ASes to participate. If an AS wants to optimize end-to-end performance,
multiple ASes would have to participate in the bidding. Lastly, a verification
mechanism would be required to ensure that the paths in fact yielded the
performance solicited.
%This framework must  A key insight is that
%BGP could allow for negotiation of routes between non-adjacent
%ASes by using a ``virtual peering'' mechanism.
%None of the proposed extensions to BGP have been successfully deployed. To ensure
%broad impact, my research will focus on ensuring a low
%barrier to widespread adoption. To begin, the proposed extensions must be
%backwards-compatible and incremental, such that any two ASes can start utilizing the extensions
%without support from intermediate ASes.
%
%Content distribution networks and providers completely rely on the
%Internet for the business, and thus have huge incentives for improving the
%performance and reliability of interdomain routes. Exploring the
%revenue opportunities for solving issues affecting these stake-holders is a
%promising avenue for ensuring adoption. 

%ASes to maintain arbitrary routing policies, and should not require
%ASes to expose unwanted information. 

\noindent{\bf Intellectual Merit:} Negotiating mutually beneficial routes between
ASes with competing interests is fundamentally challenging.

\noindent{\bf Broader Impact:} The performance and reliability of the Internet is
crucial. Providing a general framework for negotiating
interdomain routing greatly improves the overall quality of the
Internet and every application that runs on top of it.

\noindent{\bf Apologies:} I apologize for the quality of this proposal.
% Anticipated results.
% A statement attesting to the originality of the research proposal.
% Do I understand the relavent principles and techniques?
%I will use a measurement-based approach to evaluating the proposed protocol extensions;
%it is only by having a deep understanding of what the problems are to begin
%with that we can hope to have a chance of providing a permanent solution for them.
%\noindent{\bf Related Work} \\
%\indent Consensus routing is a proposed extension to BGP designed to eliminate
%transient routing loops, blackholes, and other problems caused by BGP
%convergence effects.
%
%Wiser is a framework for negotiating routing policies between adjacent ASes.
%
%How is my work different (better than) these approaches?
% backup paths, ratul's wiser stuff, brighten godfrey's various proposals.
%"Make sure to introduce the problem well, why it is important - motivate it well and
% at a high level.  Talk about why it is hard, some of the details on why it hasn't
% been solved.  Talk about your planned approach and why it is different / has a chance
% of working.  It is proposed, so you should not have all the details yet, but should
% have a framework.  Then, talk about how you will evaluate it and know if you succeeded.
%  Then perhaps close with the impact it will have.  You don't need to cite tons of
% related work in detail, but distill some of it down to show that you understand what's
% been done and what hasn't."
% Identification of /critical/ issues.
% \noindent{\bf Intellectual Merit}:
% \noindent{\bf Broader Impact}:

\noindent{\bf References}: 
\begin{packed_enum}
\item R. Krishnan, H.V. Madhyastha, S. Jain, S. Srinivasan, A.
Krishnamurthy, T. Anderson, J. Gao, ``Moving Beyond End-to-End Path
Information to Optimize CDN Performance'', IMC 2009
\item R. Mahajan, D. Wetherall, T. Anderson, ``Negotiation Based Routing
Between Neighboring Domains'', NSDI 2005
\item W. Xu, J. Rexford, ``MIRO: multi-path interdomain routing'', SIGCOMM
2006
%\item J. John, E. Katz-Bassett, A. Krishnamurthy, T. Anderson, A.
%Venkataramani, ``Consensus Routing: The Internet as a Distributed System'',
%NSDI 2008
%\item B. Quoitin, O. Bonaventure, ``A Cooperative Approach to
%Interdomain Traffic Engineering'', NGI, 2005
\end{packed_enum}

\end{document}
